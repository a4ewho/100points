\documentclass[a4paper, 12pt]{article}%тип документа

%Русский язык
\usepackage[T2A]{fontenc} %кодировка
\usepackage[utf8]{inputenc} %кодировка исходного кода
\usepackage[english,russian]{babel} %локализация и переносы

\usepackage{multirow}

%Вставка картинок
\usepackage{graphicx}
\DeclareGraphicsExtensions{.pdf,.png,.jpg}
\usepackage{float}
\usepackage{wrapfig}
\usepackage{tabularx}
\usepackage{amsmath}


%Производные
\usepackage{physics}

\setlength{\parindent}{0pt}

%Математика
\usepackage{amsmath, amsfonts, amssymb, amsthm, mathtools}

\usepackage[left=10mm, top=15mm, right=10mm, bottom=15mm, footskip=10mm]{geometry}

%Заголовок
\pagenumbering{gobble}
\title{\textbf{Как решать задание №5 с автоматами?}}
%\author{Нечаева Дарья}
\date{}

\begin{document}
    \maketitle
    %\begin{figure}[h]
	%\begin{center}
		%\includegraphics[scale = 0.5]{графики0.jpg}
	%\end{center}
    %\end{figure}

    \section*{\underline{Задание:} В торговом центре два одинаковых автомата продают кофе. Обслуживание автоматов происходит по вечерам после закрытия центра. Известно, что вероятность события «К вечеру в первом автомате закончится кофе» равна 0,25. Такая же вероятность события «К вечеру во втором автомате закончится кофе». Вероятность того, что кофе к вечеру закончится в обоих автоматах, равна 0,15. Найдите вероятность того, что к вечеру дня кофе останется в обоих автоматах.}

    В этом задании вероятности нам уже даны, поэтому произведением вероятностей мы пользоваться не можем. \newline

    Обозначим 4 случая:
    \begin{itemize}
        \item \underline{\textbf{ОО}} -- кофе осталось в обоих автоматах
        \item \underline{\textbf{ЗО}} -- кофе закончилось только в первом автомате
        \item \underline{\textbf{ОЗ}} -- кофе закончилось только во втором автомате
        \item \underline{\textbf{ЗЗ}} -- кофе закончилось в обоих автоматах
    \end{itemize}

    Из условия следует: \newline
    \underline{\textbf{ЗЗ}} = 0,15 \newline

    Кофе закончилось в первом автомате включает в себя два случая (З стоит на первом месте): \newline
    \underline{\textbf{ЗО}} и \underline{\textbf{ЗЗ}} \newline
    Значит из условия: \newline
    \underline{\textbf{ЗО}} + \underline{\textbf{ЗЗ}} = 0,25 \newline
    Находим \underline{\textbf{ЗО}}:  \newline
    \underline{\textbf{ЗО}} = 0,25 - \underline{\textbf{ЗЗ}} = 0,25 - 0,15 = 0,1 \newline

    Кофе закончилось во втором автомате включает в себя два случая (З стоит на втором месте): \newline
    \underline{\textbf{ОЗ}} и \underline{\textbf{ЗЗ}} \newline
    Значит из условия: \newline
    \underline{\textbf{ОЗ}} + \underline{\textbf{ЗЗ}} = 0,25 \newline
    Находим \underline{\textbf{ОЗ}}:  \newline
    \underline{\textbf{ОЗ}} = 0,25 - \underline{\textbf{ЗЗ}} = 0,25 - 0,15 = 0,1 \newline

    По сумме всевозможных вероятностей знаем: \newline
    \underline{\textbf{ОО}} + \underline{\textbf{ЗО}} + \underline{\textbf{ОЗ}} + \underline{\textbf{ЗЗ}} = 1 \newline
    Находим \underline{\textbf{ОО}}:  \newline 
    \underline{\textbf{ОО}} = 1 - \underline{\textbf{ЗО}} - \underline{\textbf{ОЗ}} - \underline{\textbf{ЗЗ}} = 1 - 0,1 - 0,1 - 0,15 = 0,65 \newline

    \section*{\underline{Ответ: 0,65}}
    
\end{document}
