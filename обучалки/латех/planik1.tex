\documentclass[a4paper, 12pt]{article}%тип документа

%Русский язык
\usepackage[T2A]{fontenc} %кодировка
\usepackage[utf8]{inputenc} %кодировка исходного кода
\usepackage[english,russian]{babel} %локализация и переносы

\usepackage{multirow}

%Вставка картинок
\usepackage{graphicx}
\DeclareGraphicsExtensions{.pdf,.png,.jpg}
\usepackage{float}
\usepackage{wrapfig}
\usepackage{tabularx}
\usepackage{amsmath}


%Производные
\usepackage{physics}

\setlength{\parindent}{0pt}

%Математика
\usepackage{amsmath, amsfonts, amssymb, amsthm, mathtools}

\usepackage[left=10mm, top=15mm, right=18mm, bottom=15mm, footskip=10mm]{geometry}

%Заголовок
\pagenumbering{gobble}
\title{\textbf{Планиметрия. Трек Легкой геометрии 5. Урок 20. Сложная планиметрия}}
\author{03-09.11.2025}
\date{}

\begin{document}
    \maketitle

\section{Обратная теорема Пифагора}
    \texttt{Пример, как грамотно доказать, что треугольник прямоугольный, с помощью обратной теоремы Пифагора.} \newline
    \begin{figure}[H]
	\begin{center}
		\includegraphics[scale = 0.8]{pryamoy.png}
	\end{center}
    \end{figure}
    
    // Пусть доказали, что $\angle ABC = 90^\circ$, $AB = 4$, $BC = 3$ и $AC = 5$ \newline

    $CB^2 + BC^2 = 4^2 + 3^2 = 16 + 9 = 25 = 5^2 = AC^2$ \newline
    
    Значит по обратной теореме Пифагора: \newline
    $\triangle ABC$ -- прямоугольный
    \newpage

\section{Несколько подобных треугольников}

    \texttt{Если несколько треугольников подобны между собой, это еще не значит, что их коэффициенты подобия равны.} \newline
    
    \begin{figure}[H]
	\begin{center}
		\includegraphics[scale = 0.7]{podobie.png}
	\end{center}
    \end{figure}
    
    Треугольники $\triangle AB_1C_1$, $\triangle AB_2C_2$ и $\triangle AB_3C_3$ подобны по двум углам ($\angle AB_1C_1 = \angle AB_2C_2 = \angle AB_3C_3 = 77^\circ$ и $\angle B_3AC_3$ -- общий) \newline

    Рассмотрим их попарно: \newline
    
    $\triangle AB_1C_1 \sim \triangle AB_2C_2$: \newline
    \begin{equation*}
        \frac{B_1C_1}{B_2C_2} = \frac{AC_1}{AC_2} = \frac{AB_1}{AB_2} = \frac{x}{2x} = \frac{1}{2} = k_1
    \end{equation*}

    $\triangle AB_1C_1 \sim \triangle AB_3C_3$: \newline
    \begin{equation*}
        \frac{B_1C_1}{B_3C_3} = \frac{AC_1}{AC_3} = \frac{AB_1}{AB_3} = \frac{x}{3x} = \frac{1}{3} = k_2
    \end{equation*}

    $\triangle AB_2C_2 \sim \triangle AB_3C_3$: \newline
    \begin{equation*}
        \frac{B_2C_2}{B_3C_3} = \frac{AC_2}{AC_3} = \frac{AB_2}{AB_3} = \frac{2x}{3x} = \frac{2}{3} = k_3
    \end{equation*}

    Видно, что $k_1 \neq k_2 \neq k_3$ \newline
    Значит мы не можем написать: \newline
    \begin{equation*}
        \frac{B_1C_1}{B_2C_2} = \frac{AC_1}{AC_2} = \frac{AB_1}{AB_2} \neq \frac{B_1C_1}{B_3C_3} = \frac{AC_1}{AC_3} = \frac{AB_1}{AB_3} \neq \frac{B_2C_2}{B_3C_3} = \frac{AC_2}{AC_3} = \frac{AB_2}{AB_3}
    \end{equation*}
    \newpage

\section{Разница между теоремой о пропорциональных отрезках, теоремой Фалеса и отношением из подобия треугольников}
    \texttt{Многие их путают между собой. Разберем различия.} \newline
    
    \subsection*{Теорема Фалеса}

    \textbf{\underline{Теорема Фалеса}}: Если параллельные прямые пересекают две данные прямые и отсекают на одной из них равные отрезки, то они отсекают равные отрезки и на другой данной прямой. \newline
    
    \textbf{$B_1C_1$ || $B_2C_2$ || $B_3C_3$. Если $B_1B_2 = B_2B_3$, то $C_1C_2 = C_2C_3$} 

    \begin{figure}[H]
	\begin{center}
		\includegraphics[scale = 0.4]{fales.png}
	\end{center}
    \end{figure}

    \subsection*{Теорема о пропорциональных отрезках}

    \textbf{\underline{Теорема о пропорциональных отрезках}}: Параллельные прямые, пересекающие стороны угла, отсекают от его сторон пропорциональные отрезки. \newline
    
    \textbf{$B_1C_1$ || $B_2C_2$ || $B_3C_3$. Значит}
    \begin{equation*}
        \frac{B_1B_2}{B_2B_3} = \frac{C_1C_2}{C_2C_3} 
    \end{equation*}

    \begin{figure}[H]
	\begin{center}
		\includegraphics[scale = 0.4]{prop_otrezki.png}
	\end{center}
    \end{figure}

    // По сути, теорема Фалеса является частным случаем теоремы о пропорциональных отрезках, поэтому вместо теоремы Фалеса всегда можно использовать теорему о пропорциональных отрезках.

    \subsection*{Отношение из подобия треугольников}

    \textbf{Соответствующие стороны подобных треугольников соотносятся как одно число — коэффициент подобия k} \newline
    
    \textbf{$\triangle AB_1C_1 \sim \triangle AB_2C_2$. Значит}
    \begin{equation*}
        \frac{B_1C_1}{B_2C_2} = \frac{AC_1}{AC_2} = \frac{AB_1}{AB_2} 
    \end{equation*}

    \begin{figure}[H]
	\begin{center}
		\includegraphics[scale = 0.4]{podobie2.png}
	\end{center}
    \end{figure}

    \subsection*{Разница между теоремой о пропорциональных отрезках и отношением сторон подобных треугольников}

    \begin{figure}[H]
	\begin{center}
		\includegraphics[scale = 0.4]{podobie2.png}
	\end{center}
    \end{figure}
    
    Пусть $\triangle AB_1C_1 \sim \triangle AB_2C_2$ \newline

    \textbf{По теореме о пропорциональных отрезках}
    \begin{equation*}
        \frac{AB_1}{B_1B_2} = \frac{AC_1}{C_1C_2} 
    \end{equation*}

    \textbf{Отношение сторон подобных треугольников}
    \begin{equation*}
        \frac{AB_1}{AB_2} = \frac{AC_1}{AC_2}
    \end{equation*}
    \newpage
    
\section{Пункт а) 1 задачи из дз 20 урока}

    \texttt{Приведу часть решения 1 задачи из дз 20 урока} \newline
    
    \begin{figure}[H]
	\begin{center}
		\includegraphics[scale = 0.8]{20_1.png}
	\end{center}
    \end{figure}

    \subsection*{1 способ}

    Рассмотрим $\triangle SCE$:
    \begin{equation*}
        cos(\angle ASD) = \frac{SC}{SE}
    \end{equation*}

    Рассмотрим $\triangle ASD$:
    \begin{equation*}
        cos(\angle ASD) = \frac{SA}{SD}
    \end{equation*}

    Рассмотрим $\triangle SHA$:
    \begin{equation*}
        cos(\angle ASD) = \frac{SH}{SA}
    \end{equation*}

    Рассмотрим $\triangle SBC$:
    \begin{equation*}
        cos(\angle ASD) = \frac{SB}{SC}
    \end{equation*}

    Рассмотрим равенство:
    \begin{equation*}
        \frac{SH}{SA} = \frac{SB}{SC} \;\;\;\;\;\;\;\;\;\; SA = \frac{SH \cdot SC}{SB}
    \end{equation*}

    Рассмотрим равенство:
    \begin{equation*}
        \frac{SA}{SD} = \frac{SC}{SE} \;\;\;\;\;\;\;\;\;\; \frac{SH \cdot SC}{SB \cdot SD} = \frac{SC}{SE} \;\;\;\;\;\;\;\;\;\; \frac{SH}{SD} = \frac{SC\cdot SB}{SE\cdot SC} \;\;\;\;\;\;\;\;\;\; \frac{SH}{SD} = \frac{SB}{SE}
    \end{equation*}

    $\triangle SBH \sim \triangle SED$ по двум пропорциональным сторонам и углу между ними ( $\frac{SH}{SD} = \frac{SB}{SE}$ и $\angle ASD$ -- общий)\newline
    Значит $\angle SHB = \angle SDE$, являются соответственными углами при пересечении $SD$ прямых $BH$ и $ED$, значит $BH$ || $ED$

    \subsection*{2 способ}
    $\triangle SBC \sim \triangle SAD$ по двум углам ($\angle SBC = \angle SAD = 90^\circ$ и $\angle ASD$ -- общий)\newline
    Значит \newline
    \begin{equation*}
        \frac{SB}{SA} = \frac{SC}{SD} \;\;\;\;\;\;\;\;\;\; SB = \frac{SC \cdot SA}{SD}
    \end{equation*}

    $\triangle SCE \sim \triangle SHA$ по двум углам ($\angle SCE = \angle SHA = 90^\circ$ и $\angle ASD$ -- общий)\newline
    Значит \newline
    \begin{equation*}
        \frac{SE}{SA} = \frac{SC}{SH} \;\;\;\;\;\;\;\;\;\; SE = \frac{SC \cdot SA}{SH}
    \end{equation*}

    Рассмотрим отношение: \newline
    \begin{equation*}
        \frac{SB}{SE} = \frac{\frac{SC \cdot SA}{SD}}{\frac{SC \cdot SA}{SH}} \;\;\;\;\;\;\;\;\;\; \frac{SB}{SE} = \frac{SH}{SD}
    \end{equation*}

    $\triangle SBH \sim \triangle SED$ по двум пропорциональным сторонам и углу между ними ( $\frac{SH}{SD} = \frac{SB}{SE}$ и $\angle ASD$ -- общий)\newline
    Значит $\angle SHB = \angle SDE$, являются соответственными углами при пересечении $SD$ прямых $BH$ и $ED$, значит $BH$ || $ED$
    \newpage

\section{Отрезки основания, образованные высотами, в равнобедренной трапеции}

    \texttt{Многие из вас знают, что высоты из вершин в ранобедренной трапеции образуют равные отрезки, которые можно вычислить как полуразность оснований} \newline

    \begin{equation*}
        DH = CL = \frac{CD - AB}{2}
    \end{equation*}

    \begin{figure}[H]
	\begin{center}
		\includegraphics[scale = 0.5]{rb_trapeciya.png}
	\end{center}
    \end{figure}

    \texttt{Эту формулу можно использовать без доказательства. Но именно в пункте а) 3 задачи из дз 20 урока ее нужно было доказать. Покажу, как это можно сделать.} \newline

    $ABCD$ -- равнобедренная трапеция. Проведем в трапеции на основание CD две высоты -- $AH$ и $BC$ \newline

    Рассмотрим $ABLH$: \newline
    \begin{equation*}
        AH \perp HL \;\;\;\;\;\;\;\;\;\; BL \perp HL \;\;\;\;\;\;\;\;\;\; AB \parallel HL
    \end{equation*}
    Значит $ABLH$ -- прямоугольник. Значит $AB = HL$, $AH = BL$ \newline

    Рассмотрим $\triangle ADH$ и $\triangle BCL$: \newline
    $\angle AHD = \angle BLC = 90^\circ$, значит треугольники прямоугольные \newline
    $\triangle ADH \sim \triangle BCL$ по гипотенузе и острому углу ($\angle ADH = \angle BCL$ и $AD = BC$ (т.к. $ABCD$ -- р/б трапеция)) \newline
    Значит $DH = CL$ \newline  
    \begin{equation*}
        CD = CL + LH + HD = AB + 2DH \;\;\;\;\;\;\;\;\;\; 2DH = CD - AB \;\;\;\;\;\;\;\;\;\; CL = DH = \frac{CD - AB}{2}
    \end{equation*}

    

    

    
    
\end{document}